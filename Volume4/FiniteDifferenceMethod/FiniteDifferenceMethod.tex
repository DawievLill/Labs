\lab{The Finite Difference Method}{The Finite Difference Method}
\label{lab:finitedifference2}

A \textbf{finite difference} for a function $f(x)$ is an expression of the form $f(x+s)-f(x+t)$.
Finite differences can give a good approximation of derivatives.

Suppose we have a function $u(x)$, defined on an interval $[a,b]$.
Let $a = x_0, x_1, \ldots x_{n-1}, x_n=b$ be a grid of $n+1$ evenly spaced points, with $x_{i+1}-x_i = h$, where $h = (b-a)/n$.

You are used to seeing the derivative $u'(x)$, which can written in centered-difference form as:
\begin{align*}
u'(x)=\lim_{h\to\infty}\frac{u(x+h)-u(x-h)}{2h}.
 \end{align*}
Suppose we are interested in knowing the value of the derivative at the points $\{x_i\}$. Even if we don't have a formula for $u'(x)$, we can approximate it using finite differences.
We first write the Taylor polynomial expansion of $u(x+h)$ and $u(x-h)$ centered at $x$. This gives

\begin{align}
u(x+h)=u(x)+u'(x)h+\frac{1}{2}u''(x)h^2+\frac{1}{6}u'''(x)h^3+\mathcal{O}(h^4)\label{plus}\\
u(x-h)=u(x)-u'(x)h+\frac{1}{2}u''(x)h^2-\frac{1}{6}u'''(x)h^3+\mathcal{O}(h^4)\label{minus}
\end{align}
Subtracting \eqref{minus} from \eqref{plus} and rearranging gives
\begin{align*}
u'(x)=\frac{u(x+h)-u(x-h)}{2h}+\mathcal{O}(h^2).
\end{align*}

In terms of our grid points $\{x_i\}$, we have:
\begin{align*}
u'(x_i) \approx \frac{u(x_i+h)-u(x_i-h)}{2h}=\frac{u(x_{i+1})-u(x_{i-1})}{2h}.
\end{align*}
We won't worry about the derivative at the endpoints, $u'(x_0)$ and $u'(x_n)$.
This allows us to approximate the values $\{u'(x_i)\}$ as the solution to a system of equations:

\begin{align}
\frac{1}{2h}
\begin{array}{c}
\begin{bmatrix}
-1 &0 & 1 & \\
& -1 & 0 & 1  \\
& & \ddots & \ddots & \ddots & \\
& & & -1 & 0 & 1 \\
& & & & -1 & 0 & 1
\end{bmatrix}\\
{\scriptstyle\textcolor{red}{(n-1)\times(n+1)}}
\end{array}\cdot
\begin{array}{c}
\begin{bmatrix}
u(x_0) \\ u(x_1)\\ \vdots  \\ u(x_{n-1}) \\ u(x_n)
\end{bmatrix}\\
{\scriptstyle\textcolor{red}{(n+1)\times1}}
\end{array} \approx
\begin{array}{c}
\begin{bmatrix}
u'(x_1) \\ u'(x_2)\\ \vdots  \\ u'(x_{n-2}) \\ u'(x_{n-1})
\end{bmatrix}\\
{\scriptstyle\textcolor{red}{(n-1)\times1}}
\end{array}.
\label{finitedifference2:up}
\end{align}

\noindent This can be rewritten with a $(n-1)\times(n-1)$ tridiagonal matrix instead:
\begin{align}
\frac{1}{2h}
\begin{array}{c}
\begin{bmatrix}
0 & 1 & \\
-1 & 0 & 1  \\
& \ddots & \ddots & \ddots \\
& & -1 & 0 & 1 \\
& & & -1 & 0
\end{bmatrix}\\
{\scriptstyle\textcolor{red}{(n-1)\times(n-1)}}
\end{array}\cdot
\begin{array}{c}
\begin{bmatrix}
u(x_1) \\ u(x_2)\\ \vdots  \\ u(x_{n-2}) \\ u(x_{n-1})
\end{bmatrix}\\
{\scriptstyle\textcolor{red}{(n-1)\times1}}
\end{array} +
\begin{array}{c}
\begin{bmatrix}
-u(x_0)/(2h) \\ 0 \\ \vdots  \\ 0 \\ u(x_n)/(2h)
\end{bmatrix}\\
{\scriptstyle\textcolor{red}{(n-1)\times1}}
\end{array} \approx
\begin{array}{c}
\begin{bmatrix}
u'(x_1) \\ u'(x_2)\\ \vdots  \\ u'(x_{n-2}) \\ u'(x_{n-1})
\end{bmatrix}\\
{\scriptstyle\textcolor{red}{(n-1)\times1}}
\end{array}.
\label{finitedifference2:u}
\end{align}

\noindent Next, we will consider the approximation for $u''(x)$.
If we let
\[u'(x) \approx \frac{u(x+\frac{h}{2})-u(x-\frac{h}{2})}{h}\]
then
\[u''(x) \approx \frac{u'(x+\frac{h}{2})-u'(x-\frac{h}{2})}{h} \approx \frac{\frac{u((x+\frac{h}{2})+\frac{h}{2})-u((x+\frac{h}{2})-\frac{h}{2})}{h}-\frac{u((x-\frac{h}{2})+\frac{h}{2})-u((x-\frac{h}{2})-\frac{h}{2})}{h}}{h}\]
\[= \frac{u(x+h) - 2u(x) + u(x-h)}{h^2}\]
You can achieve the same result by again consider the Taylor polynomial expansion and adding \eqref{plus} and \eqref{minus} and rearranging.
Thus
\[u''(x_i) \approx \frac{u(x_i+h) - 2u(x_i) + u(x_i-h)}{h^2}= \frac{u(x_{i+1}) - 2u(x_i) + u(x_{i-1})}{h^2}\]
Again ignoring the second derivative at the endpoints, this can be written in matrix form as
\begin{align}
\frac{1}{h^2}
\begin{array}{c}
\begin{bmatrix}
1 & -2 & 1 & \\
& 1 & -2 & 1  \\
& &\ddots &\ddots & \ddots  \\
& & & 1 & -2 & 1 \\
& & & & 1 & -2 & 1
\end{bmatrix}\\
{\scriptstyle\textcolor{red}{(n-1)\times(n+1)}}
\end{array}\cdot
\begin{array}{c}
\begin{bmatrix}
u(x_0) \\ u(x_1)\\ \vdots  \\ u(x_{n-1}) \\ u(x_n)
\end{bmatrix}\\
{\scriptstyle\textcolor{red}{(n+1)\times1}}
\end{array}
 \approx
 \begin{array}{c}
\begin{bmatrix}
u''(x_1) \\ u''(x_2)\\ \vdots  \\ u''(x_{n-2}) \\ u''(x_{n-1})
\end{bmatrix}\\
{\scriptstyle\textcolor{red}{(n-1)\times1}}
\label{finitedifference2:uppp}
\end{array}.
\end{align}

\noindent This can also be written with a $(n-1)\times(n-1)$ tridiagonal matrix:
\begin{align}
\frac{1}{h^2}
\begin{array}{c}
\begin{bmatrix}
-2 & 1 & \\
1 & -2 & 1  \\
& \ddots & \ddots & \ddots \\
& & 1 & -2 & 1 \\
& & & 1 & -2
\end{bmatrix}\\
{\scriptstyle\textcolor{red}{(n-1)\times(n-1)}}
\end{array}\cdot
\begin{array}{c}
\begin{bmatrix}
u(x_1) \\ u(x_2)\\ \vdots  \\ u(x_{n-2}) \\ u(x_{n-1})
\end{bmatrix}\\
{\scriptstyle\textcolor{red}{(n-1)\times1}}
\end{array} +
\begin{array}{c}
\begin{bmatrix}
u(x_0)/h^2 \\ 0 \\ \vdots  \\ 0 \\ u(x_n)/h^2
\end{bmatrix}\\
{\scriptstyle\textcolor{red}{(n-1)\times1}}
 \end{array} =
 \begin{array}{c}
\begin{bmatrix}
u''(x_1) \\ u''(x_2)\\ \vdots  \\ u''(x_{n-2}) \\ u''(x_{n-1})
\end{bmatrix}\\
{\scriptstyle\textcolor{red}{(n-1)\times1}}
\label{finitedifference2:upp}
\end{array}
\end{align}

\begin{problem}
	Let $u(x) = \sin((x+\pi)^2-1)$. Use \eqref{finitedifference2:up} - \eqref{finitedifference2:upp} to approximate $\frac{1}{2}u'' - u'$ at the grid points where $a=0$, $b=1$, and $n=10$. Graph the result.
\end{problem}

The previous equations are not only useful for approximating derivatives, but they can be also used to solve differential equations. Suppose that instead of knowing the function $u(x)$, we know that $\frac{1}{2}u'' - u' = f$, where the function $f(x)$ is given. How do we solve for $u(x)$?

\section*{Finite Difference Methods}

Numerical methods for differential equations seek to approximate the exact solution $u(x)$ at some finite collection of points in the domain of the problem.
Instead of analytically solving the original differential equation, defined over an infinite-dimensional function space, they use a well-chosen finite system of algebraic equations to approximate the original problem.

Consider the following differential equation:
\begin{align}
	\begin{split}
	&{ } \epsilon u''(x)-u(x)'= f(x), \quad x \in (0,1), \\
	&{ } u(0) = \alpha, \quad u(1) = \beta.
	\end{split} \label{eqn:singular_perturbed_BVP}
\end{align}
Equation \eqref{eqn:singular_perturbed_BVP} can be written $D u = f,$ where $D = \epsilon \frac{d^2}{dx^2} - \frac{d}{dx}$ is a differential operator defined on the infinite-dimensional space of functions that are twice continuously differentiable on $[0,1]$ and satisfy $u(0) = \alpha$, $u(1) = \beta$.

We look for an approximate solution $\{U_i\}$, where \[U_i \approx u(x_i)\] on an evenly spaced grid of points, $a = x_0, x_1, \ldots, x_n = b$,.
Our finite difference method will replace the differential operator $D = \epsilon \frac{d^2}{dx^2} - \frac{d}{dx}$, (which is defined on an infinite-dimensional space), with finite difference operators (defined on a finite dimensional space).
To do this, we replace derivative terms in the differential equation with appropriate difference expressions.

Recalling that
\begin{align*}
\frac{d^2}{dx^2} u(x_i) &= \frac{u(x_{i+1})- 2u(x_i) + u(x_{i-1})}{h^2} + \mathcal{O}(h^2),\\
\frac{d}{dx} u(x_i) &= \frac{u(x_{i+1})-u(x_{i-1})}{2h} + \mathcal{O}(h^2).
\end{align*}
we define the finite difference operator $D_h$ by
\begin{align}
D_h U_i = \epsilon \frac{1}{h^2}\left(U_{i+1} -2U_i + U_{i-1}\right) - \frac{1}{2h} \left(U_{i+1}-U_{i-1} \right). \label{fd_operator}
\end{align}
Thus we discretize equation \eqref{eqn:singular_perturbed_BVP} using the equations
\begin{align*}
	\frac{\epsilon}{h^2} (U_{i+1}- 2U_i + U_{i-1}) - \frac{1}{2h} \left(U_{i+1}-U_{i-1} \right)  &= f(x_i), \quad i = 1, \ldots, n-1,
\end{align*}
along with boundary conditions $U_{0} = \alpha$, $U_{n} = \beta$.

This gives $n+1$ equations and $n+1$ unknowns, and can be written in matrix form as
\[ \frac{1}{h^2}
\begin{array}{c}
\begin{bmatrix}h^2 & 0 &0&\hdots &0 \\ (\epsilon + h/2) &-2\epsilon & (\epsilon - h/2) &\hdots &0\\ \vdots &  & \ddots & &\vdots \\
0 & \hdots & (\epsilon + h/2) &-2\epsilon & (\epsilon - h/2) \\ 0 & \hdots & & 0 & h^2
\end{bmatrix}\\
{\scriptstyle\textcolor{red}{(n+1)\times(n+1)}}
\end{array}\cdot
\begin{array}{c}
\begin{bmatrix}U_0\\U_1\\ \vdots \\U_{n-1} \\U_{n}\end{bmatrix}\\
{\scriptstyle\textcolor{red}{(n+1)\times1}}
\end{array}=
\begin{array}{c}
\begin{bmatrix}\alpha\\f(x_1)\\ \vdots \\ f(x_{n-1}) \\ \beta \end{bmatrix}\\
{\scriptstyle\textcolor{red}{(n+1)\times1}}
\end{array}.
\]
As before, we can remove two equations to modify the system to obtain an $(n-1)\times (n-1)$ tridiagonal system:
\begin{align}
    \label{eqn:practical_fin_dif}
    \begin{split}
\frac{1}{h^2}
\begin{array}{c}
\begin{bmatrix}-2\epsilon & (\epsilon - h/2) &0&\hdots &0 \\ (\epsilon + h/2) &-2\epsilon & (\epsilon - h/2) &\hdots &0\\ \vdots &  & \ddots & &\vdots \\
0 & \hdots & (\epsilon + h/2) &-2\epsilon & (\epsilon - h/2) \\ 0 & \hdots & & (\epsilon + h/2) &-2\epsilon
\end{bmatrix}\\
{\scriptstyle\textcolor{red}{(n-1)\times(n-1)}}
\end{array}
\cdot
\begin{array}{c}
\begin{bmatrix}U_1\\U_2\\ \vdots \\U_{n-2} \\U_{n-1}\end{bmatrix}\\
{\scriptstyle\textcolor{red}{(n-1)\times1}}
\end{array}\\
 =
 \begin{array}{c}
 \begin{bmatrix}f(x_1) -\alpha(\epsilon + h/2)/h^2 \\f(x_2)\\ \vdots \\ f(x_{n-2})\\ f(x_{n-1})-\beta(\epsilon - h/2)/h^2 \end{bmatrix}\\
 {\scriptstyle\textcolor{red}{(n-1)\times1}}
 \end{array}.
    \end{split}
\end{align}



\begin{problem}
Use equation \eqref{eqn:practical_fin_dif} to solve the singularly perturbed BVP \eqref{eqn:singular_perturbed_BVP} with $\epsilon = 1/10$, $f(x) = -1$, $\alpha=1$, and $\beta=3$ on a grid with $n=30$ subintervals. Graph the solution.
This BVP is called singularly perturbed because of the location of the parameter $\epsilon$.
For $\epsilon = 0$ the ODE has a drastically different character - it then becomes first order, and can no longer support two boundary conditions.
    \label{prob:finitedifference2:prob1}
\end{problem}

\begin{figure}[h]
\centering
\includegraphics[width=10cm]{figure2.pdf}
\caption{The solution to Problem \ref{prob:finitedifference2:prob1}.
The solution gets steeper near $x = 1$ as $\epsilon $ gets small.}
\end{figure}


\begin{comment}
\begin{lstlisting}
import numpy as np
from scipy.sparse import spdiags
from scipy.sparse.linalg import spsolve
import matplotlib.pyplot as plt

def bvp(func, epsilon, alpha, beta, N):
	a,b = 0., 1. 	# Interval for the BVP
	h = (b-a)/N 	# The length of each subinterval

	# Initialize and define the vector F on the right
	F = np.empty(N-1.)
	F[0] = func(a+1.*h)-alpha*(epsilon+h/2.)*h**(-2.)
	F[N-2] = func(a+(N-1)*h)-beta*(epsilon-h/2.)*h**(-2.)
	for j in xrange(1,N-2):
		F[j] = func(a + (j+1)*h)

	# Here we define the arrays that will go on the diagonals of A
	data = np.empty((3,N-1))
	data[0,:] = -2.*epsilon*np.ones((1,N-1)) # main diagonal
	data[1,:]  = (epsilon+h/2.)*np.ones((1,N-1))  	 # off-diagonals
	data[2,:] = (epsilon-h/2.)*np.ones((1,N-1))
	# Next we specify on which diagonals they will be placed, and create A
	diags = np.array([0,-1,1])
	A=h**(-2.)*spdiags(data,diags,N-1,N-1).asformat('csr')

	U = np.empty(N+1)
	U[1:-1] = spsolve(A,F)
	U[0], U[-1] = alpha, beta
	return np.linspace(a,b,N+1), U

x, y = bvp(lambda x:-1., epsilon=.05,alpha=1, beta=3, N=400)
plt.plot(x,y,'-k',linewidth=2.0)
plt.show()

\end{lstlisting}
\end{comment}


\section*{A heuristic test for convergence}
\begin{figure}[ht]
\centering
\includegraphics[width=12cm]{example_convergence.pdf}
\caption{Demonstration of second order convergence for the finite difference approximation \eqref{fd_operator} of the BVP given in \eqref{eqn:singular_perturbed_BVP} with $\epsilon = .5$. } \label{fig:finitedifference2}
\end{figure}

The finite differences used above are second order approximations of the first and second derivatives of a function.  It seems reasonable to expect that the numerical solution would converge at a rate of about $\mathcal{O}(h^2)$. How can we check that a numerical approximation is reasonable?

Suppose a finite difference method is $\mathcal{O}(h^p)$ accurate.
This means that the error $E(h) \approx Ch^p$ for some constant $C$ as $h \to 0$ (in other words, for $h>0$ small enough).

So compute the approximation $y_k$ for each stepsize $h_k$, $h_1 > h_2> \ldots>h_m$.
$y_m$ should be the most accurate approximation, and will be thought of as the true solution.
Then the error of the approximation for
stepsize $h_k, k < m,$ is
\begin{align*}
	E(h_k) &= \max( \abs{ y_k - y_m}) \approx C h_k^p ,\\
	\log(E(h_k)) &= \log(C) + p \log(h_k).
\end{align*}

Thus on a log-log plot of $E(h)$ vs. $h$, these values should be on a straight line with slope $p$ when $h$ is small enough to start getting convergence. We should note that demonstrating second-order convergence does NOT imply that the numerical approximation is converging to the correct solution.


\begin{problem}
Visualize the $\mathcal{O}(h^2)$ convergence of this finite difference method by producing a loglog plot similar to Figure \ref{fig:finitedifference2}, except in the case $\epsilon=.1$. Implement a function \li{singular_bvp} to compute the finite difference solution to \ref{eqn:singular_perturbed_BVP}. Using $n = 5 \times 2^0,5 \times 2^1, \dots ,5 \times 2^9$ subintervals, compute 10 approximate solutions. 


To produce the plot, treat the approximation with $n = 5 \times 2^9$ subintervals as the "true solution", and measure the error for the other approximations against it. Note that, since the number of subintervals for each approximation is a multiple of 2, we can compute the $L_{\infty}$ error for the $n=5 \times 2^j$ approximation by using the \li{step} argument in the array slicing syntax:
\begin{lstlisting}
# best approximation
sol_best = singular_bvp(eps,alpha,beta,f,5*(2**9))

# approximation with 5*(2^j) intervals
sol_approx = singular_bvp(eps,alpha,beta,f,5*(2**j))

# approximation error
error = np.max(np.abs(sol_approx - sol_best[::2**(9-j)]))
\end{lstlisting}
\end{problem}

\begin{problem}
Extend your finite difference code to the case of a general second order linear BVP with boundary conditions:
\begin{align*}
	&{ } a_1(x)y'' +a_2(x)y'+ a_3(x) y = f(x), \quad x \in (a,b),\\
	&{ } y(a) = \alpha, \quad y(b) = \beta.
\end{align*}
Use your code to solve the boundary value problem
\begin{align*}
	\epsilon y'' - 4(\pi - x^2)y = \cos x, \\
	y(0) = 0, \quad y(\pi/2) = 1,
\end{align*}
for $\epsilon = 0.1$ on a grid with $n=30$ subintervals.
\label{prob:finitedifference2:prob3}
Be sure to modify the finite difference operator $D_h$ in \eqref{fd_operator} correctly.

\end{problem}



\begin{figure}[h]
\centering
\includegraphics[width=10cm]{figure3.pdf}
\caption{The solution to Problem \ref{prob:finitedifference2:prob3}.
% The solution gets steeper near $x = 1$ as $\epsilon $ gets small.
}
\end{figure}


The next few problems will help you test your finite difference code.

\begin{problem}
Numerically solve the boundary value problem
\begin{align*}
	\epsilon y'' + xy' = -\epsilon \pi^2 \cos(\pi x) - \pi x \sin(\pi x), \\
	y(-1) = -2, \quad y(1) = 0,
\end{align*}
for $\epsilon = 0.1, 0.01,$ and $0.001$. Use a grid with $n=150$ subintervals.
\label{prob:finitedifference2:prob4}
\end{problem}

\begin{problem}
Numerically solve the boundary value problem
\begin{align*}
	(\epsilon +x^2)y'' + 4xy' + 2y = 0, \\
	y(-1) = 1/(1+\epsilon), \quad y(1) = 1/(1+\epsilon),
\end{align*}
for $\epsilon = 0.05, 0.02$. Use a grid with $n=150$ subintervals.
\label{prob:finitedifference2:prob5}
\end{problem}

\begin{figure}[h]
\centering
\includegraphics[width=11cm]{figure4.pdf}
\caption{The solution to Problem \ref{prob:finitedifference2:prob4}.
}
\end{figure}

\begin{figure}
\centering
\includegraphics[width=11cm]{figure5.pdf}
\caption{The solution to Problem \ref{prob:finitedifference2:prob5}.
}
\end{figure}
